
\documentclass[11pt]{article}

\usepackage{a4wide}
\usepackage{graphicx}
\usepackage{listings}
\usepackage{xcolor}
\usepackage{hyperref}

\setlength{\parindent}{0pt}

\definecolor{codegreen}{rgb}{0,0.6,0}
\definecolor{codegray}{rgb}{0.5,0.5,0.5}
\definecolor{codepurple}{rgb}{0.58,0,0.82}
\definecolor{backcolour}{rgb}{0.95,0.95,0.92}

\lstset{
    basicstyle=\ttfamily,      % 设置基本字体
    breaklines=true,           % 自动换行
    backgroundcolor=\color{backcolour},   
    commentstyle=\color{codegreen},
    keywordstyle=\color{magenta},
    numberstyle=\tiny\color{codegray},
    stringstyle=\color{codepurple},
    basicstyle=\ttfamily\footnotesize,
    breakatwhitespace=false,         
    breaklines=true,                 
    captionpos=b,                    
    keepspaces=true,                 
    numbers=left,                    
    numbersep=5pt,                  
    showspaces=false,                
    showstringspaces=false,
    showtabs=false,                  
    tabsize=2
}

\begin{document}

\title{Assignment3 Team6 Report}
\author{Haohan Fu, Jiayang Xu, Xibin Yu, Xi Wang, Zhengyang Cheng and Haoyu Ju}
\date{26, March, 2025}
\maketitle

您的目标是:

1. 使用逆向工程工具分析恶意软件的代码,找出它在做什么。
2. 确定恶意软件的目标文件/目录。
3. 恢复恶意软件用来锁定目标文件/目录的密钥/密码。
4. 汉克备份了最重要的文件,但这些文件已经被勒索软件加密。使用恢复的密钥/密码,编程一个能够解密汉克文件的工具。

我们只要求简明扼要地提供必要的信息。需要考虑的重要方面:结构、适当的详细程度、清晰度、可复制性、完整性。

\section{Analyze the malware's code}
\subsection{Start}
We used Ghidra to analyze the malware code. We found it difficult to find the code that implements the encryption function directly from the entry point, so we started at defined strings in the program. Then we found the AES encryption function AES\_Encrypt\_140007080, whose function call tree is shown in Figure \ref{Call Trees}:

\subsection{The AES encryption function}
\textbf{AES\_Encrypt\_140007080} is the AES encryption function.
\begin{figure}[htbp]
    \centering
    \includegraphics[width=0.6\textwidth]{img/Call Trees.png}
    \caption{Function call trees}
    \label{Call Trees}
\end{figure}



\textbf{AES-CBC-128}



\section{Determine what files are targeted}


\textbf{EncryptAndRenameFiles\_140007590}
\begin{lstlisting}[language=c++, caption=EncryptAndRenameFiles\_140007590]
    do
    {
      /* Exclude Directory */
      if ((local_10e8.dwFileAttributes & 0x10) == 0)
      {
        CopyWPath_140007bc0(local_a68, 0x104, dir);
        input_addr = local_a68;
        ConcatWPath_140007b20(input_addr, 0x104, local_10e8.cFileName);
        GetModuleFileNameW((HMODULE)0x0, local_858, 0x104);
        thunk_FUN_14000c700((undefined8 *)local_e98,
                            (undefined8 *)local_10e8.cFileName, 6);
        /* Exclude "~en" */
        iVar2 = wcscmp(local_e98, L"~en");
        if (iVar2 != 0)
        {
          _Str2 = PathFindFileNameW(local_858);
          /* Exclude Malware Itelf */
          iVar2 = wcscmp(local_10e8.cFileName, _Str2);
          if (iVar2 != 0)
          {
            CopyWPath_140007bc0(local_648, 0x104, dir);
            output_addr = local_648;
            ConcatWPath_140007b20(output_addr, 0x104, (short *)&DAT_140070fd8);
            ConcatWPath_140007b20(output_addr, 0x104, local_10e8.cFileName);
            AES_Encrypt_140007080(input_addr, output_addr);
            DeleteFileW(input_addr);
          }
        }
      }
      BVar3 = FindNextFileW(local_1110, &local_10e8);
    } while (BVar3 != 0);
\end{lstlisting}

\textbf{RansomwareProcessor\_140008240}
\begin{lstlisting}[language=c++, caption=RansomwareProcessor\_140008240]
void RansomwareProcessor_140008240(void)
{
    /*...*/
    WCHAR dir[264];
    /*...*/
    printf((char *)L"Getting current directory. ");
    GetCurrentDirectoryW(0x104, dir);
    EncryptAndRenameFiles_140007590(dir);
    Sleep(10000);
    /*...*/
}
\end{lstlisting}




\section{Recover the AES key}
As noted above, the memory address of the AES key is the second parameter of the InitEncryption\_140008790 function:
\begin{lstlisting}[language=c++, caption=call of InitEncryption\_140008790]
                                            /*Address of AES key*/
InitEncryption_140008790((longlong)context_array,0x140086000,(undefined8 *)IV_140086010);
\end{lstlisting}

Then we took a screenshot of the key in Ghidra, as shown in Figure \ref{fig:key}.

The AES key is '8d02e65e508308dd743f0dd4d31e484d'.
\begin{figure}[htbp]
    \centering
    \includegraphics[width=0.8\textwidth]{img/key.png}
    \caption{the AES key in Ghidra}
    \label{fig:key}
\end{figure}

\section{Decrypt Hank's files.}
The tool to decrypt Hank's files is 'assinment3-team6-data/AES\_decrypt.py'.

There are two important functions in the program.

\subsection{Decrypt a block}
Only keep the actual plaintext length portion, the rest is meaningless padding used during encryption.

\begin{lstlisting}[language=python, caption=decrypt\_block]
def decrypt_block(ciphertext_block, key, iv, actual_plaintext_len):
    cipher = AES.new(key, AES.MODE_CBC, iv)
    decrypted_block = cipher.decrypt(ciphertext_block)
    return decrypted_block[:actual_plaintext_len]
\end{lstlisting}

\subsection{Decrypt the file}

\begin{lstlisting}[language=python, caption=decrypt\_file]
def decrypt_file(input_path, output_path, key_hex):
    #...
    with open(input_path, 'rb') as f_in, open(output_path, 'wb') as f_out:
        while True:
            # Read the 16-byte IV
            iv = f_in.read(16)  #0a0b0c0d0e0fa0b0c0d0e0f0aabbccdd
            #...
            # Read the 4-byte actual plaintext length
            block_len_bytes = f_in.read(4)
            #...
            actual_plaintext_len = struct.unpack('<I', block_len_bytes)[0]
            # Read the encrypted 1008-byte block
            ciphertext_block = f_in.read(BLOCK_SIZE)
            #...
            # AES Decrypt
            plaintext_block = decrypt_block(ciphertext_block, key, iv, actual_plaintext_len)
            f_out.write(plaintext_block)
            #...
\end{lstlisting}

To use this python script, please install pycryptodome firstly.
\begin{lstlisting}
pip3 install pycryptodome
\end{lstlisting}

Then replace the following line with YOUR directory of the files to be decrypted, and DO NOT add a '/' to the end of your directory.
\begin{lstlisting}[language=python]
FILE_DIRECTORY = "HanksBackup"
\end{lstlisting}


\section*{Academic Conduct \& Plagiarism:}
We take plagiarism seriously. By submitting your solution, you agree that:

1. The submission is your group's own work and that you have not worked with others in preparing this assignment.

2. Your submitted solutions and report were written by you and **in your own words**, except for any materials from published or other sources which are clearly indicated and acknowledged as such by appropriate referencing.

3. The work is not copied from any other person's work (published or unpublished), web site, book or other source, and has not previously been submitted for assessment either at the University of Birmingham or elsewhere.

4. You have not asked, or paid, others to prepare any part of this work.

\newpage
\appendix
\section{Ghidra screenshots}
% Screenshot of the key in Ghidra is shown in Figure \ref{fig:key}. The key is '8d02e65e508308dd743f0dd4d31e484d'.
% \begin{figure}[htbp]
%     \centering
%     \includegraphics[width=0.8\textwidth]{img/key.png}
%     \caption{the key in Ghidra}
%     \label{fig:key}
% \end{figure}

Screenshot of IV in Ghidra is shown in Figure \ref{fig:IV}. IV is '0a0b0c0d0e0fa0b0c0d0e0f0aabbccdd'.
\begin{figure}[htbp]
    \centering
    \includegraphics[width=0.8\textwidth]{img/IV.png}
    \caption{IV in Ghidra}
    \label{fig:IV}
\end{figure}

\section{C-style decompiled codes}
All the C-style decompiled codes mentioned above can be found in the
directory\\'assinment3-team6-data/C-style decompiled code'.

In addition, there are some functions not mentioned above, but which are also
valuable (because they are part of the function call tree), listed below:

\begin{enumerate}
    \item entry.c \begin{lstlisting}[language=C++, caption=entry.c]
void entry(void)
{
    __security_init_cookie();
    RansomwareEntryPoint_14000afe0();
    return;
}
\end{lstlisting}

    \item RansomwareEntryPoint.c \begin{lstlisting}[language=C++, caption=RansomwareEntryPoint\_14000afe0.c]
ulonglong RansomwareEntryPoint_14000afe0(void)
{
    /*...*/
            __scrt_get_show_window_mode();
            _get_wide_winmain_command_line();
            /*Ransomware Processor here*/
            uVar3 = RansomwareProcessor();
            uVar7 = __scrt_is_managed_app();
    /*...*/
}
\end{lstlisting}

\end{enumerate}

\section{Decrypted files}
Decrypted files can be found at:

\url{https://github.com/Superior-Josh/FMPT-Assignment3/tree/main/HanksBackup_decrypted}

\subsection{Screenshot of the output}
The successful output of the decryption tool is shown in Figure \ref{fig:output}.
\begin{figure}[htbp]
    \centering
    \includegraphics[width=0.8\textwidth]{img/output.png}
    \caption{Decryption tool output}
    \label{fig:output}
\end{figure}

A decrypted example (SampleVideo\_1280\texttimes 720\_30mb.mp4) is shown in
Figure \ref{fig:decrypted_exapmle}.
\begin{figure}[htbp]
    \centering
    \includegraphics[width=0.8\textwidth]{img/decrypted_exapmle.png}
    \caption{Decrypted file example}
    \label{fig:decrypted_exapmle}
\end{figure}

\end{document}